\documentclass{article}
\usepackage[left=2cm,right=2cm,top=3cm,bottom=3cm,letterpaper]{geometry}
\usepackage[spanish]{babel}
\usepackage[utf8]{inputenc}
\usepackage{amsmath}
\usepackage{amsfonts}
\usepackage{amssymb}
\usepackage{amsthm}
\usepackage{longtable} 
\usepackage{listings}
\usepackage{tabularx}
\usepackage{alltt}
\lstset{
  basicstyle=\ttfamily,
  mathescape
}

\title{Lenguajes de Programación 2016-1\\Tarea 3}
\author{Ricardo Garcia Garcia \and  Juan Carlos López López \and Luis Rodrigo Rojo Morales}
\date{\today\\ Facultad de Ciencias UNAM}
  
\begin{document}
\maketitle
\section*{Problema I}
Haga el juicio de tipo para la función fibonacci y el predicado empty?\\

\begin{lstlisting}
(define fibonacci
  $\boxed{1}$(lambda (n)
    $\boxed{2}$(cond
      [$\boxed{3}$(<= n 2) $\boxed{4}$1]
      [$\boxed{5}$else $\boxed{6}$(+ $\boxed{7}$(fibonacci $\boxed{8}$(- n 1)) $\boxed{9}$(fibonacci $\boxed{10}$(- n 2)))])))
\end{lstlisting}

\begin{center}
 \begin{longtable}{ | l | p{10 cm} | p{5 cm} | }
  \hline
  Acción & Stack & Sustitución \\ \hline \hline
  Inicio & [$\boxed{1}$] = [n] $\rightarrow$ [$\boxed{2}$] 									& Vacio	\\
	 & [$\boxed{3}$] = boolean		     										&   	\\
	 & [$<=$] = [n] $\rightarrow$ [2] $\rightarrow$ [$\boxed{3}$] = 
	 number $\rightarrow$ number $\rightarrow$ boolean									& 	\\ 
	 & [$\boxed{4}$] = number												&	\\ 
	 & [else] = [$\boxed{5}$] = [$\boxed{6}$]										&	\\
	 & [+] = [$\boxed{7}$] $\rightarrow$ [$\boxed{9}$] $\rightarrow$ [$\boxed{6}$] = 
	 number $\rightarrow$ number $\rightarrow$ number									&	\\ 
	 & [$\boxed{1}$] = [$\boxed{8}$] $\rightarrow$ [$\boxed{7}$]								&	\\
	 & [-] = [n] $\rightarrow$ [1] $\rightarrow$ [$\boxed{8}$] = [n] $\rightarrow$ [2] $\rightarrow$ [$\boxed{10}$] =
	   number $\rightarrow$ number $\rightarrow$ number									&	\\
	 & [$\boxed{1}$] = [$\boxed{10}$] $\rightarrow$ [$\boxed{9}$] 								&	\\ 
	 & [$\boxed{2}$] = [$\boxed{4}$] = [$\boxed{6}$]									&	\\ \hline
	 
  Paso 3 & [$\boxed{3}$] = boolean		     										& [$\boxed{1}$] $\mapsto$ [n] $\rightarrow$ [$\boxed{2}$]  \\
	 & [$<=$] = [n] $\rightarrow$ [2] $\rightarrow$ [$\boxed{3}$] = 
	 number $\rightarrow$ number $\rightarrow$ boolean									& 	\\ 
	 & [$\boxed{4}$] = number												&	\\ 
	 & [else] = [$\boxed{5}$] = [$\boxed{6}$]										&	\\
	 & [+] = [$\boxed{7}$] $\rightarrow$ [$\boxed{9}$] $\rightarrow$ [$\boxed{6}$] = 
	 number $\rightarrow$ number $\rightarrow$ number									&	\\ 
	 & [n] $\rightarrow$ [$\boxed{2}$] = [$\boxed{8}$] $\rightarrow$ [$\boxed{7}$]						&	\\
	 & [-] = [n] $\rightarrow$ [1] $\rightarrow$ [$\boxed{8}$] = [n] $\rightarrow$ [2] $\rightarrow$ [$\boxed{10}$] =
	   number $\rightarrow$ number $\rightarrow$ number									&	\\
	 & [n] $\rightarrow$ [$\boxed{2}$] = [$\boxed{10}$] $\rightarrow$ [$\boxed{9}$] 					&	\\ 
	 & [$\boxed{2}$] = [$\boxed{4}$] = [$\boxed{6}$]									&	\\ \hline
	 
  Paso 3  & [$<=$] = [n] $\rightarrow$ [2] $\rightarrow$ boolean = 
	 number $\rightarrow$ number $\rightarrow$ boolean									& [$\boxed{1}$] $\mapsto$ [n] $\rightarrow$ [$\boxed{2}$]	\\ 
	 & [$\boxed{4}$] = number												& [$\boxed{3}$] $\mapsto$ boolean	\\ 
	 & [else] = [$\boxed{5}$] = [$\boxed{6}$]										&	\\
	 & [+] = [$\boxed{7}$] $\rightarrow$ [$\boxed{9}$] $\rightarrow$ [$\boxed{6}$] = 
	 number $\rightarrow$ number $\rightarrow$ number									&	\\ 
	 & [n] $\rightarrow$ [$\boxed{2}$] = [$\boxed{8}$] $\rightarrow$ [$\boxed{7}$]						&	\\
	 & [-] = [n] $\rightarrow$ [1] $\rightarrow$ [$\boxed{8}$] = [n] $\rightarrow$ [2] $\rightarrow$ [$\boxed{10}$] =
	   number $\rightarrow$ number $\rightarrow$ number									&	\\
	 & [n] $\rightarrow$ [$\boxed{2}$] = [$\boxed{10}$] $\rightarrow$ [$\boxed{9}$] 					&	\\ 
	 & [$\boxed{2}$] = [$\boxed{4}$] = [$\boxed{6}$]									&	\\ \hline
	 
  Paso 5 & [n] = number 													& [$\boxed{1}$] $\mapsto$ [n] $\rightarrow$ [$\boxed{2}$]	\\ 
	 & [2] = number														& [$\boxed{3}$] $\mapsto$ boolean \\
	 & [$\boxed{4}$] = number												& 	\\ 
	 & [else] = [$\boxed{5}$] = [$\boxed{6}$]										&	\\
	 & [+] = [$\boxed{7}$] $\rightarrow$ [$\boxed{9}$] $\rightarrow$ [$\boxed{6}$] = 
	 number $\rightarrow$ number $\rightarrow$ number									&	\\ 
	 & [n] $\rightarrow$ [$\boxed{2}$] = [$\boxed{8}$] $\rightarrow$ [$\boxed{7}$]						&	\\
	 & [-] = [n] $\rightarrow$ [1] $\rightarrow$ [$\boxed{8}$] = [n] $\rightarrow$ [2] $\rightarrow$ [$\boxed{10}$] =
	   number $\rightarrow$ number $\rightarrow$ number									&	\\
	 & [n] $\rightarrow$ [$\boxed{2}$] = [$\boxed{10}$] $\rightarrow$ [$\boxed{9}$] 					&	\\ 
	 & [$\boxed{2}$] = [$\boxed{4}$] = [$\boxed{6}$]									&	\\ \hline
	 
  Paso 3 & [2] = number														& [$\boxed{1}$] $\mapsto$ number $\rightarrow$ [$\boxed{2}$] \\
	 & [$\boxed{4}$] = number												& [$\boxed{3}$] $\mapsto$ boolean	\\ 
	 & [else] = [$\boxed{5}$] = [$\boxed{6}$]										& [n] $\mapsto$ number	\\
	 & [+] = [$\boxed{7}$] $\rightarrow$ [$\boxed{9}$] $\rightarrow$ [$\boxed{6}$] = 
	 number $\rightarrow$ number $\rightarrow$ number									&	\\ 
	 & number $\rightarrow$ [$\boxed{2}$] = [$\boxed{8}$] $\rightarrow$ [$\boxed{7}$]					&	\\
	 & [-] = number $\rightarrow$ [1] $\rightarrow$ [$\boxed{8}$] = number $\rightarrow$ [2] $\rightarrow$ [$\boxed{10}$] =
	   number $\rightarrow$ number $\rightarrow$ number									&	\\
	 & number $\rightarrow$ [$\boxed{2}$] = [$\boxed{10}$] $\rightarrow$ [$\boxed{9}$] 					&	\\ 
	 & [$\boxed{2}$] = [$\boxed{4}$] = [$\boxed{6}$]									&	\\ \hline
	 
 Paso 3  & [$\boxed{4}$] = number												& [$\boxed{1}$] $\mapsto$ number $\rightarrow$ [$\boxed{2}$]	\\ 
	 & [else] = [$\boxed{5}$] = [$\boxed{6}$]										& [$\boxed{3}$] $\mapsto$ boolean	\\
	 & [+] = [$\boxed{7}$] $\rightarrow$ [$\boxed{9}$] $\rightarrow$ [$\boxed{6}$] = 
	 number $\rightarrow$ number $\rightarrow$ number									& [n] $\mapsto$ number	\\ 
	 & number $\rightarrow$ [$\boxed{2}$] = [$\boxed{8}$] $\rightarrow$ [$\boxed{7}$]					& [2] $\mapsto$ number	\\
	 & [-] = number $\rightarrow$ [1] $\rightarrow$ [$\boxed{8}$] = number $\rightarrow$ number $\rightarrow$ [$\boxed{10}$] =
	   number $\rightarrow$ number $\rightarrow$ number									&	\\
	 & number $\rightarrow$ [$\boxed{2}$] = [$\boxed{10}$] $\rightarrow$ [$\boxed{9}$] 					&	\\ 
	 & [$\boxed{2}$] = [$\boxed{4}$] = [$\boxed{6}$]									&	\\ \hline
	 
 Paso 3  & [else] = [$\boxed{5}$] = [$\boxed{6}$]										& [$\boxed{1}$] $\mapsto$ number $\rightarrow$ [$\boxed{2}$]	\\
	 & [+] = [$\boxed{7}$] $\rightarrow$ [$\boxed{9}$] $\rightarrow$ [$\boxed{6}$] = 
	 number $\rightarrow$ number $\rightarrow$ number									& [$\boxed{3}$] $\mapsto$ boolean	\\ 
	 & number $\rightarrow$ [$\boxed{2}$] = [$\boxed{8}$] $\rightarrow$ [$\boxed{7}$]					& [n] $\mapsto$ number	\\
	 & [-] = number $\rightarrow$ [1] $\rightarrow$ [$\boxed{8}$] = number $\rightarrow$ number $\rightarrow$ [$\boxed{10}$] =
	   number $\rightarrow$ number $\rightarrow$ number									& [2] $\mapsto$ number	\\
	 & number $\rightarrow$ [$\boxed{2}$] = [$\boxed{10}$] $\rightarrow$ [$\boxed{9}$] 					& [$\boxed{4}$] $\mapsto$ number	\\ 
	 & [$\boxed{2}$] = number = [$\boxed{6}$]										&	\\ \hline
	 
 Paso 3  & [+] = [$\boxed{7}$] $\rightarrow$ [$\boxed{9}$] $\rightarrow$ [$\boxed{6}$] = 
	 number $\rightarrow$ number $\rightarrow$ number									& [$\boxed{1}$] $\mapsto$ number $\rightarrow$ [$\boxed{2}$]	\\ 
	 & number $\rightarrow$ [$\boxed{2}$] = [$\boxed{8}$] $\rightarrow$ [$\boxed{7}$]					& [$\boxed{3}$] $\mapsto$ boolean	\\
	 & [-] = number $\rightarrow$ [1] $\rightarrow$ [$\boxed{8}$] = number $\rightarrow$ number $\rightarrow$ [$\boxed{10}$] =
	   number $\rightarrow$ number $\rightarrow$ number									& [n] $\mapsto$ number	\\
	 & number $\rightarrow$ [$\boxed{2}$] = [$\boxed{10}$] $\rightarrow$ [$\boxed{9}$] 					& [2] $\mapsto$ number	\\ 
	 & [$\boxed{2}$] = number = [$\boxed{6}$]										& [$\boxed{4}$] $\mapsto$ number	\\ 
	 &	& [$\boxed{5}$] $\mapsto$ [$\boxed{6}$] \\ \hline
	 
 Paso 5  & [$\boxed{7}$] = number												& [$\boxed{1}$] $\mapsto$ number $\rightarrow$ [$\boxed{2}$] \\
	 & [$\boxed{9}$] = number												& [$\boxed{3}$] $\mapsto$ boolean \\
	 & [$\boxed{6}$] = number												& [n] $\mapsto$ number	\\ 
	 & number $\rightarrow$ [$\boxed{2}$] = [$\boxed{8}$] $\rightarrow$ [$\boxed{7}$]					& [2] $\mapsto$ number	\\
	 & [-] = number $\rightarrow$ [1] $\rightarrow$ [$\boxed{8}$] = number $\rightarrow$ number $\rightarrow$ [$\boxed{10}$] =
	   number $\rightarrow$ number $\rightarrow$ number									& [$\boxed{4}$] $\mapsto$ number	\\
	 & number $\rightarrow$ [$\boxed{2}$] = [$\boxed{10}$] $\rightarrow$ [$\boxed{9}$] 					& [$\boxed{5}$] $\mapsto$ [$\boxed{6}$]	\\ 
	 & [$\boxed{2}$] = number = [$\boxed{6}$]										& 	\\ \hline
	 
 Paso 3  & [$\boxed{9}$] = number												& [$\boxed{1}$] $\mapsto$ number $\rightarrow$ [$\boxed{2}$] \\
	 & [$\boxed{6}$] = number												& [$\boxed{3}$] $\mapsto$ boolean	\\ 
	 & number $\rightarrow$ [$\boxed{2}$] = [$\boxed{8}$] $\rightarrow$ number						& [n] $\mapsto$ number	\\
	 & [-] = number $\rightarrow$ [1] $\rightarrow$ [$\boxed{8}$] = number $\rightarrow$ number $\rightarrow$ [$\boxed{10}$] =
	   number $\rightarrow$ number $\rightarrow$ number									& [2] $\mapsto$ number	\\
	 & number $\rightarrow$ [$\boxed{2}$] = [$\boxed{10}$] $\rightarrow$ [$\boxed{9}$] 					& [$\boxed{4}$] $\mapsto$ number	\\ 
	 & [$\boxed{2}$] = number = [$\boxed{6}$]										& [$\boxed{5}$] $\mapsto$ [$\boxed{6}$]	\\ 
	 &	& [$\boxed{7}$] $\mapsto$ number \\ \hline
	 
 Paso 3  & [$\boxed{6}$] = number												& [$\boxed{1}$] $\mapsto$ number $\rightarrow$ [$\boxed{2}$] \\ 
	 & number $\rightarrow$ [$\boxed{2}$] = [$\boxed{8}$] $\rightarrow$ number						& [$\boxed{3}$] $\mapsto$ boolean	\\
	 & [-] = number $\rightarrow$ [1] $\rightarrow$ [$\boxed{8}$] = number $\rightarrow$ number $\rightarrow$ [$\boxed{10}$] =
	   number $\rightarrow$ number $\rightarrow$ number									& [n] $\mapsto$ number	\\
	 & number $\rightarrow$ [$\boxed{2}$] = [$\boxed{10}$] $\rightarrow$ number	 					& [2] $\mapsto$ number	\\ 
	 & [$\boxed{2}$] = number = [$\boxed{6}$]										& [$\boxed{4}$] $\mapsto$ number	\\ 
	 &	& [$\boxed{5}$] $\mapsto$ [$\boxed{6}$] \\
	 &	& [$\boxed{7}$] $\mapsto$ number \\ 
	 &	& [$\boxed{9}$] $\mapsto$ number \\ \hline
	 
 Paso 3  & number $\rightarrow$ [$\boxed{2}$] = [$\boxed{8}$] $\rightarrow$ number						& [$\boxed{1}$] $\mapsto$ number $\rightarrow$ [$\boxed{2}$]	\\
	 & [-] = number $\rightarrow$ [1] $\rightarrow$ [$\boxed{8}$] = number $\rightarrow$ number $\rightarrow$ [$\boxed{10}$] =
	   number $\rightarrow$ number $\rightarrow$ number									& [$\boxed{3}$] $\mapsto$ boolean	\\
	 & number $\rightarrow$ [$\boxed{2}$] = [$\boxed{10}$] $\rightarrow$ number	 					& [n] $\mapsto$ number	\\ 
	 & [$\boxed{2}$] = number = number											& [2] $\mapsto$ number	\\ 
	 &	& [$\boxed{4}$] $\mapsto$ number \\
	 &	& [$\boxed{5}$] $\mapsto$ number \\
	 &	& [$\boxed{7}$] $\mapsto$ number \\ 
	 &	& [$\boxed{9}$] $\mapsto$ number \\ 
	 &	& [$\boxed{6}$] $\mapsto$ number \\ \hline
	 
 Paso 5  & number = [$\boxed{8}$]						 						& [$\boxed{1}$] $\mapsto$ number $\rightarrow$ [$\boxed{2}$]	\\
	 & [$\boxed{2}$] = number												& [$\boxed{3}$] $\mapsto$ boolean \\
	 & [-] = number $\rightarrow$ [1] $\rightarrow$ [$\boxed{8}$] = number $\rightarrow$ number $\rightarrow$ [$\boxed{10}$] =
	   number $\rightarrow$ number $\rightarrow$ number									& [n] $\mapsto$ number	\\
	 & number $\rightarrow$ [$\boxed{2}$] = [$\boxed{10}$] $\rightarrow$ number	 					& [2] $\mapsto$ number	\\ 
	 & [$\boxed{2}$] = number = number											& [$\boxed{4}$] $\mapsto$ number	\\ 
	 &	& [$\boxed{5}$] $\mapsto$ number \\
	 &	& [$\boxed{7}$] $\mapsto$ number \\ 
	 &	& [$\boxed{9}$] $\mapsto$ number \\ 
	 &	& [$\boxed{6}$] $\mapsto$ number \\ \hline
 
 Paso 4  & [$\boxed{2}$] = number												& [$\boxed{1}$] $\mapsto$ number $\rightarrow$ [$\boxed{2}$] \\
	 & [-] = number $\rightarrow$ [1] $\rightarrow$ number = number $\rightarrow$ number $\rightarrow$ [$\boxed{10}$] =
	   number $\rightarrow$ number $\rightarrow$ number									& [$\boxed{3}$] $\mapsto$ boolean	\\
	 & number $\rightarrow$ [$\boxed{2}$] = [$\boxed{10}$] $\rightarrow$ number	 					& [n] $\mapsto$ number	\\ 
	 & [$\boxed{2}$] = number = number											& [2] $\mapsto$ number \\ 
	 &	& [$\boxed{4}$] $\mapsto$ number \\
	 &	& [$\boxed{5}$] $\mapsto$ number \\
	 &	& [$\boxed{7}$] $\mapsto$ number \\ 
	 &	& [$\boxed{9}$] $\mapsto$ number \\ 
	 &	& [$\boxed{6}$] $\mapsto$ number \\ 
	 &	& [$\boxed{8}$] $\mapsto$ number \\ \hline
	 
 Paso 3  & [-] = number $\rightarrow$ [1] $\rightarrow$ number = number $\rightarrow$ number $\rightarrow$ [$\boxed{10}$] =
	   number $\rightarrow$ number $\rightarrow$ number								& [$\boxed{1}$] $\mapsto$ number $\rightarrow$ number	\\
	 & number $\rightarrow$ number = [$\boxed{10}$] $\rightarrow$ number	 					& [$\boxed{3}$] $\mapsto$ boolean	\\ 
	 & number = number = number											& [n] $\mapsto$ number \\ 
	 &	& [2] $\mapsto$ number \\
	 &	& [$\boxed{4}$] $\mapsto$ number \\
	 &	& [$\boxed{5}$] $\mapsto$ number \\
	 &	& [$\boxed{7}$] $\mapsto$ number \\ 
	 &	& [$\boxed{9}$] $\mapsto$ number \\ 
	 &	& [$\boxed{6}$] $\mapsto$ number \\ 
	 &	& [$\boxed{8}$] $\mapsto$ number \\ 
	 &	& [$\boxed{2}$] $\mapsto$ number \\ \hline
	 
 Paso 5  & [1] = number								& [$\boxed{1}$] $\mapsto$ number $\rightarrow$ number	\\
	 & number = [$\boxed{10}$]						& [$\boxed{3}$] $\mapsto$ boolean \\
	 & number $\rightarrow$ number = [$\boxed{10}$] $\rightarrow$ number	& [n] $\mapsto$ number	\\ 
	 & number = number = number						& [2] $\mapsto$ number \\ 
	 &	& [$\boxed{4}$] $\mapsto$ number \\
	 &	& [$\boxed{5}$] $\mapsto$ number \\
	 &	& [$\boxed{7}$] $\mapsto$ number \\ 
	 &	& [$\boxed{9}$] $\mapsto$ number \\ 
	 &	& [$\boxed{6}$] $\mapsto$ number \\ 
	 &	& [$\boxed{8}$] $\mapsto$ number \\ 
	 &	& [$\boxed{2}$] $\mapsto$ number \\ \hline
	 
 Paso 3  & number = [$\boxed{10}$]						& [$\boxed{1}$] $\mapsto$ number $\rightarrow$ number \\
	 & number $\rightarrow$ number = [$\boxed{10}$] $\rightarrow$ number	& [$\boxed{3}$] $\mapsto$ boolean \\ 
	 & number = number = number						& [n] $\mapsto$ number \\ 
	 &	& [2] $\mapsto$ number \\
	 &	& [$\boxed{4}$] $\mapsto$ number \\
	 &	& [$\boxed{5}$] $\mapsto$ number \\
	 &	& [$\boxed{7}$] $\mapsto$ number \\ 
	 &	& [$\boxed{9}$] $\mapsto$ number \\ 
	 &	& [$\boxed{6}$] $\mapsto$ number \\ 
	 &	& [$\boxed{8}$] $\mapsto$ number \\ 
	 &	& [$\boxed{2}$] $\mapsto$ number \\ 
	 &	& [1] $\mapsto$ number \\ \hline
	 
 Paso 4  & number $\rightarrow$ number = number $\rightarrow$ number	& [$\boxed{1}$] $\mapsto$ number $\rightarrow$ number \\ 
	 & number = number = number					& [$\boxed{3}$] $\mapsto$ boolean \\ 
	 &	& [n] $\mapsto$ number \\
	 &	& [2] $\mapsto$ number \\
	 &	& [$\boxed{4}$] $\mapsto$ number \\
	 &	& [$\boxed{5}$] $\mapsto$ number \\
	 &	& [$\boxed{7}$] $\mapsto$ number \\ 
	 &	& [$\boxed{9}$] $\mapsto$ number \\ 
	 &	& [$\boxed{6}$] $\mapsto$ number \\ 
	 &	& [$\boxed{8}$] $\mapsto$ number \\ 
	 &	& [$\boxed{2}$] $\mapsto$ number \\ 
	 &	& [1] $\mapsto$ number \\ 
	 &	& [$\boxed{10}$] $\mapsto$ number \\ \hline
	 
 Paso 1  & number = number = number					& [$\boxed{1}$] $\mapsto$ number $\rightarrow$ number \\ 
	 &	& [$\boxed{3}$] $\mapsto$ boolean \\
	 &	& [n] $\mapsto$ number \\
	 &	& [2] $\mapsto$ number \\
	 &	& [$\boxed{4}$] $\mapsto$ number \\
	 &	& [$\boxed{5}$] $\mapsto$ number \\
	 &	& [$\boxed{7}$] $\mapsto$ number \\ 
	 &	& [$\boxed{9}$] $\mapsto$ number \\ 
	 &	& [$\boxed{6}$] $\mapsto$ number \\ 
	 &	& [$\boxed{8}$] $\mapsto$ number \\ 
	 &	& [$\boxed{2}$] $\mapsto$ number \\ 
	 &	& [1] $\mapsto$ number \\ 
	 &	& [$\boxed{10}$] $\mapsto$ number \\ \hline
	 
 Paso 1  & vacio& [$\boxed{1}$] $\mapsto$ number $\rightarrow$ number \\ 
	 &	& [$\boxed{3}$] $\mapsto$ boolean \\
	 &	& [n] $\mapsto$ number \\
	 &	& [2] $\mapsto$ number \\
	 &	& [$\boxed{4}$] $\mapsto$ number \\
	 &	& [$\boxed{5}$] $\mapsto$ number \\
	 &	& [$\boxed{7}$] $\mapsto$ number \\ 
	 &	& [$\boxed{9}$] $\mapsto$ number \\ 
	 &	& [$\boxed{6}$] $\mapsto$ number \\ 
	 &	& [$\boxed{8}$] $\mapsto$ number \\ 
	 &	& [$\boxed{2}$] $\mapsto$ number \\ 
	 &	& [1] $\mapsto$ number \\ 
	 &	& [$\boxed{10}$] $\mapsto$ number \\ \hline
 \end{longtable}

\end{center}

\begin{lstlisting}
 (define Empty?
  (lambda (l)
    (if (= (length l) 0)  true false)))
\end{lstlisting}

\begin{center}
 \begin{longtable}{c c c c}
 $\surd$ &	&	& \\ \cline{1-1}
 $\Gamma \vdash (l) : list$ &	&  & \\ \cline{1-1}
 $\Gamma (length (list \gets num))$ & $\surd$ & & \\ \cline{1-2}
 $\Gamma \vdash (length  \quad l):number$ & $\Gamma \vdash 0:number$ & $\surd$ & $\surd$ \\ \hline
 \multicolumn{2}{c}{$\Gamma \vdash (=(length  \quad l) \quad 0):bool$} & $\Gamma \vdash true:bool$ & $\Gamma \vdash false:bool$ \\ \hline
 \multicolumn{4}{c}{$\Gamma [l \gets list] \vdash (if...):bool$} \\ \hline
 \multicolumn{4}{c}{$\Gamma \vdash (empty?:bool) \vdash \{(lambda(l)):list \quad if...\}:bool$} \\

\end{longtable}

\end{center}

\section*{Problema II}
Considera el siguiente programa:

\begin{verbatim}
(+ 1 (first (cons true empty)))
\end{verbatim}

Este programa tiene un error de tipos.

Genera restricciones para este programa. Aísla el conjunto mas pequeño de
estas restricciones tal que, resultas juntas, identifiquen el error de tipos.

Siéntete libre de etiquetar las sub-expresiones del programa con superíndices
para usarlos cuando escribas y resuelvas tus restricciones.

\textbf {Respuesta: }
\\
$\boxed{1}$(+ $\boxed{2}$1  $\boxed{3}$ (first $\boxed{4}$ (cons $\boxed{5}$ true $\boxed{6}$ empty)))

\textbf {Conjunto de Restricciones: }
\\
Si e1 y e2 expresiones cuales quiera dentro del lenguaje entonces:
 \\
 1-[(+ e1 e2)]= number si [e1]= number y [e2]= number
 \\
 2-Si e es un numero entonces [e]= number
 \\
 3-[(first e)]= number si [e]= nlist
 \\
 4-[(cons e1 e2)]= nlist si [e1]= number y [e2]= nlist
 \\
 5- si e=true o e=false entonces [e]=boolean 
 \\
\textbf{Inferencia: }
  \\
  Tenemos que: 
  \\
  A)Para [$\boxed{1}$] =[(+ 1 (first (cons true empty)))] = number si [1] = number y [(first (cons true empty))]= number (1)\\ 
  B)Para [$\boxed{2}$] = [1] number (2)\\
  C)Para [$\boxed{3}$] = [(first (cons true empty))]= number si [(cons true empty)]=nlist (3)\\
  D)Para [$\boxed{4}$] = [(cons true empty)]=nlist si [true]= number y [empty] = nlist (4)\\
  E)Para [$\boxed{5}$] = [true]!=number sino que [true] = boolean (5)\\
  Por E) tenemos que el programa no puede continuar debido al error de tipos dadas las reglas del conjunto de restricciones
    
  

\section*{Problema III}
Considera la siguiente expresión con tipos:

\begin{verbatim}
{fun {f : C1 } : C2
  {fun {x : C3 } : C4
    {fun {y : C5 } : C6
      {cons x {f {f y}}}}}}
\end{verbatim}

Dejamos los tipos sin especificar (Cn) para que sean llenados por el proceso
de inferencia de tipos. Deriva restricciones de tipos para el programa anterior.
Luego resuelve estas restricciones. A partir de estas soluciones, rellena los
valores de las Cn. Asegúrate de mostrar todos los pasos especificados por los
algoritmos (i.e., escribir la respuesta basándose en la intuición o el conocimiento
es insuficiente). Deberás usar variables de tipo cuando sea necesario.
Para no escribir tanto, puedes etiquetar cada expresión con una variable de tipos
apropiada, y presentar el resto del algoritmo en términos solamente de estas
variables de tipos.

\begin{alltt}
\{ \boxed{1} fun \{f\}
  \{ \boxed{2} fun \{x\}
    \{ \boxed{3} fun \{y\}
      \{ \boxed{4} cons \boxed{5} x  \boxed{6} \{f \{f y\}\}\}\}\}\}
\end{alltt}

Si e1 y e2 expresiones cuales quiera dentro del lenguaje entonces:\\
 1.- [(cons e1 e2)]= nlist si [e1]= number y [e2]= nlist\\

Restricciones:\\

[$\boxed{1}$] = [f] $\rightarrow$ [$\boxed{1}$] \\

[$\boxed{2}$] = [x] $\rightarrow$ [$\boxed{2}$] \\

[$\boxed{3}$] = [y] $\rightarrow$ [$\boxed{3}$] \\

[$\boxed{4}$] = [\{cons x \{f \{f y\}\}\}] = nlist si [x] = number y [\{f \{f y\}\}] = nlist \\


Gracias a estas restricciones podemos sacar los tipos C1...C6
\begin{alltt}
\{fun \{f : number \} : nlist
  \{fun \{x : number \} : nlist
    \{fun \{y : number \} : nlist
      \{cons x \{f \{f y\}\}\}\}\}\}
\end{alltt}


\section*{Problema IV}
Considera los juicios de tipos discutidos en clase para un lenguaje glotón
(en el capitulo de \textbf{Juicios de Tipos} del libro de Shriram).
Considera ahora la versión perezosa del lenguaje. Pon especial atención a
las reglas de tipado para:

\begin{itemize}
\item definición de funciones
\item aplicación de funciones
\end{itemize}

Para cada una de estas, si crees que la regla original no cambia, explica por que no
(Si crees que ninguna de las dos cambia, puedes responder las dos partes juntas).
Si crees que algún otro juicio de tipos debe cambiar, menciónalo también.

\section*{Problema V}
¿Cuáles son las ventajas y desventajas de tener polimorfismo explícito e implícito
en los lenguajes de programación?
\\
\textbf{Respuesta: }
\\


\begin{center}
Polimorfismo Explicito
\end{center}
\begin{tabularx}{\textwidth}{X|X}
	  \textbf{Ventajas} & \textbf{Desventajas} \\
	\hline
	-Permitir el reciclamiento de código. & -Poca legibilidad de código debido al hecho de poder tener diversas funciones con el mismo nombre.
\end{tabularx}

\begin{center}
Polimorfismo Implicito
\end{center}
\begin{tabularx}{\textwidth}{X|X}
	  \textbf{Ventajas} & \textbf{Desventajas} \\
	\hline
	-? & -?
\end{tabularx}


\section*{Problema VI}
Da las ventajas y desventajas de tener lenguajes de dominio especifico (DSL)
y de propósito general. También da al menos tres ejemplos de lenguajes DSL,
cada ejemplo debe indicar el propósito del DSL y un ejemplo documentando su uso.
\\
\begin{center}
Lenguajes de Dominio Especifico (DSL)
\end{center}
\begin{tabularx}{\textwidth}{X|X}
	  \textbf{Ventajas} & \textbf{Desventajas} \\
	\hline
	-Proporciona apropiadas abstracciones y anotaciones. & -Aprenderlo para que solo pueda resolver un problema espepecifico.\\
	-Nos permiten seguridad en nivel de dominio, mientras los metodos del lenguaje esten seguros esto nos permitira seguridad cada vez que los usemos. & -Encontrar, ajustar o mantener un alcance adecuado.\\
	-Es mas sencillo desarrollar programas en un area en especifico para programadores que no sean expertos en ella. & -Gente no experta en el lenguaje no puede modificar o crear codigo facilmente.\\
\end{tabularx}

\begin{center}
Lenguajes de Propisito General (GPL)
\end{center}
\begin{tabularx}{\textwidth}{X|X}
	  \textbf{Ventajas} & \textbf{Desventajas} \\
	\hline
	-Nos ayuda a resolver problemas de diferentes areas. & -?.\\
	-?. & -?.\\
	-?. & -?.\\
\end{tabularx}
\begin{center}
Ejemplos DSL:
\end{center}

\begin{enumerate}
  \item SQL: Este lenguaje fue creado para acceder a bases de datos relacionales facilmente.\\
  		Ejemplo:\\  
		\begin{verbatim}
  		SELECT * FROM TABLA; 
  		\end{verbatim}
  		nos da toda la info alamacenada en TABLA.
  \item XML: es un lenguaje de marcas desarrollado por el World Wide Web Consortium (W3C) utilizado para almacenar datos en forma legible.\\
  		Ejemplo:
  		\begin{verbatim}
	  		<?xml version="1.0"?>
				<catalog>
				   <book id="bk101">
				      <author>Gambardella, Matthew</author>
				      <title>XML Developer's Guide</title>
				    </book>
				</catalog>
		\end{verbatim}
		Almacena datos de un libro poniendo marcas al autor y titulo.
  \item CSS: es un lenguaje usado para definir y crear la presentación de un documento estructurado escrito en HTML o XML
  Ejemplo:
		\begin{verbatim}
	  		p {
			    text-align: center;
			    color: red;
			} 
		\end{verbatim}
		Alinea la etiqueta p y le pone color rojo.
\end{enumerate}


\end{document}









