\documentclass{article}
\usepackage[left=2cm,right=2cm,top=3cm,bottom=3cm,letterpaper]{geometry}
\usepackage[spanish]{babel}
\usepackage[utf8]{inputenc}
\usepackage{amsmath}
\usepackage{amsfonts}
\usepackage{amssymb}
\usepackage{amsthm}
\usepackage{listings}% http://ctan.org/pkg/listings
\lstset{
  basicstyle=\ttfamily,
  mathescape
}

\title{Lenguajes de Programación 2016-1\\Tarea 3}
\author{Ricardo Garcia Garcia \and  Juan Carlos López López \and Luis Rodrigo Rojo Morales}
\date{\today\\ Facultad de Ciencias UNAM}
  
\begin{document}
\maketitle
\section*{Problema I}
Haga el juicio de tipo para la función fibonacci y el predicado empty?\\

\begin{lstlisting}
(define fibonacci
  $\boxed{1}$(lambda (n)
    $\boxed{2}$(cond
      [$\boxed{3}$(<= n 2) $\boxed{4}$1]
      [$\boxed{5}$else $\boxed{6}$(+ $\boxed{7}$(fibonacci $\boxed{8}$(- n 1)) $\boxed{9}$(fibonacci $\boxed{10}$(- n 2)))])))
\end{lstlisting}

\begin{center}
 \begin{tabular}{ | l | c | r | }
  \hline
  Accion & Stack & Sustitución \\ \hline \hline
  Inicio & [$\boxed{1}$] = [n] $\rightarrow$ [$\boxed{2}$] 									& Vacio	\\
	 & [$\boxed{3}$] = boolean		     										&   	\\
	 & [$<=$] = [n] $\rightarrow$ [2] $\rightarrow$ [$\boxed{3}$] = 
	 number $\rightarrow$ number $\rightarrow$ boolean									& 	\\ 
	 & [$\boxed{4}$] = number												&	\\ 
	 & [else] = [$\boxed{6}$] $\rightarrow$ [$\boxed{5}$]									&	\\
	 & [+] = [$\boxed{7}$] $\rightarrow$ [$\boxed{9}$] $\rightarrow$ [$\boxed{6}$] = 
	 number $\rightarrow$ number $\rightarrow$ number									&	\\ 
	 & [$\boxed{1}$] = [$\boxed{8}$] $\rightarrow$ [$\boxed{7}$]								&	\\
	 & [-] = [n] $\rightarrow$ [1] $\rightarrow$ [$\boxed{8}$] = [n] $\rightarrow$ [2] $\rightarrow$ [$\boxed{10}$] =
	   number $\rightarrow$ number $\rightarrow$ number									&	\\
	 & [$\boxed{1}$] = [$\boxed{10}$] $\rightarrow$ [$\boxed{9}$] 								&	\\ 
	 & [$\boxed{2}$] = [$\boxed{4}$] = [$\boxed{6}$]									&	\\ \hline
 \end{tabular}

\end{center}

\section*{Problema II}
Considera el siguiente programa:

\begin{verbatim}
(+ 1 (first (cons true empty)))
\end{verbatim}

Este programa tiene un error de tipos.

Genera restricciones para este programa. Aísla el conjunto mas pequeño de
estas restricciones tal que, resultas juntas, identifiquen el error de tipos.

Siéntete libre de etiquetar las sub-expresiones del programa con superíndices
para usarlos cuando escribas y resuelvas tus restricciones.

\section*{Problema III}
Considera la siguiente expresión con tipos:

\begin{verbatim}
{fun {f : C1 } : C2
  {fun {x : C3 } : C4
    {fun {y : C5 } : C6
      {cons x {f {f y}}}}}}
\end{verbatim}

Dejamos los tipos sin especificar (Cn) para que sean llenados por el proceso
de inferencia de tipos. Deriva restricciones de tipos para el programa anterior.
Luego resuelve estas restricciones. A partir de estas soluciones, rellena los
valores de las Cn. Asegúrate de mostrar todos los pasos especificados por los
algoritmos (i.e., escribir la respuesta basándose en la intuición o el conocimiento
es insuficiente). Deberás usar variables de tipo cuando sea necesario.
Para no escribir tanto, puedes etiquetar cada expresión con una variable de tipos
apropiada, y presentar el resto del algoritmo en términos solamente de estas
variables de tipos.

\section*{Problema IV}
Considera los juicios de tipos discutidos en clase para un lenguaje glotón
(en el capitulo de \textbf{Juicios de Tipos} del libro de Shriram).
Considera ahora la versión perezosa del lenguaje. Pon especial atención a
las reglas de tipado para:

\begin{itemize}
\item definición de funciones
\item aplicación de funciones
\end{itemize}

Para cada una de estas, si crees que la regla original no cambia, explica por que no
(Si crees que ninguna de las dos cambia, puedes responder las dos partes juntas).
Si crees que algún otro juicio de tipos debe cambiar, menciónalo también.

\section*{Problema V}
¿Cuáles son las ventajas y desventajas de tener polimorfismo explícito e implícito
en los lenguajes de programación?

\section*{Problema VI}
Da las ventajas y desventajas de tener lenguajes de dominio especifico (DSL)
y de propósito general. También da al menos tres ejemplos de lenguajes DSL,
cada ejemplo debe indicar el propósito del DSL y un ejemplo documentando su uso.
\end{document}